\documentclass[classiclight]{CV}

%................................................................
%
%                               Config
%_______________________________________________________________

\def\ProfilEN
{
    I am a creative person, passionate about flight systems, looking to work within a dynamic team. Always ready to learn new things and take initiative, I am keen to participate in a project that brings together many fields of engineering.

    Currently finishing my end-of-studies internship, I will be available from the 
    \textbf{beginning of September 2024}.
}
\def\ProfilFR
{
    Je suis une personne créative, passionnée par l'aéronautique et l'exploration spatiale, souhaitant travailler avec une équipe dynamique. Toujours prêt à apprendre de nouvelles choses et à prendre des initiatives, je suis désireux de participer à un projet qui regroupe de nombreux domaines de l'ingénierie.

    Terminant actuellement mon stage de fin d'études, je serais disponible à partir de \textbf{début septembre 2024}.
}

\ifdefined\gen
    \frfalse
    \enfalse
    \photofalse
    \headertrue
    \CompagnyNamefalse
    \Verbosefalse
    \AcademicProjectsfalse
\else
    \headertrue
    \def\langseten{}
    %\def\usephoto
    \AvailableLocfalse
    \def\CurrentSubject{SUBJECT_EMBEDDED} % SUBJECT_GRAPHICS SUBJECT_EMBEDDED
    \Verbosefalse
    \AcademicProjectstrue
\fi

\ifdefined\langsetfr
    \frtrue
\fi

\ifdefined\langseten
    \entrue
\fi

\ifdefined\usephoto
    \phototrue
\fi

\ifdefined\verbosemode
    \Verbosetrue
\fi

%................................................................
%
%                               Setup
%_______________________________________________________________
% A4 => 210 * 297
\newgeometry{
    total={200mm,282mm},
    left=5mm,
    top=10mm,
}

%................................................................
%
%                               Begin
%_______________________________________________________________
\title{CV}
\author{Sacha BELLIER}
\date{2023}

\pagestyle{empty}
\begin{document}

%................................................................
%
%                               Header
%_______________________________________________________________
\ifheader
    \section*{Start}
        \simpleheader{Sacha}{BELLIER}
        {
            \cvheadertext
            {
                Master 2 - Ingénieur Informatique \whenSubjectEmbedded{Embarquée}\whenSubjectGraphics{Bas-niveau}
            }
            {
                Master 2 - \whenSubjectEmbedded{Embedded}\whenSubjectGraphics{Low-Level} Software Engineer
            }
        }
    
    \vspace{1.4cm}
\fi

%................................................................
%
%                               Paracol Setup
%_______________________________________________________________
\setlength{\columnsep}{1.5cm}
\columnratio{0.3}[0.7]
\begin{paracol}{2}
\hbadness10000
\paracolbackgroundoptions


%................................................................
%
%                               Paracol -- Photo
%_______________________________________________________________

\small

\begin{center}
    \cvphotor{Assets/BELLIER_Sacha_Photo_CV.jpg}
    \ifphoto\bigskip\fi
\end{center}

\begin{flushleft}

%................................................................
%
%                               Paracol -- Infos
%_______________________________________________________________

\infobox{
        \cvboxlang
        {Informations personnelles}
        {Personal Information}
    }\\[0.5em]

    \iconl{\faUser} \cvsidelang{Sacha BELLIER}{Sacha BELLIER} \newline
    
    \medskip
    
    \iconl{\faAt}\small         \cvsidelang{bellier.sacha2@gmail.com}{bellier.sacha2@gmail.com}\newline
    \iconl{\faLinkedin}\small   \cvsidelang{\linktext{https://linkedin.com/in/sacha-bellier}{sacha-bellier}}{\linktext{https://linkedin.com/in/sacha-bellier}{sacha-bellier}} \newline
    \iconl{\faPhone}\small      \cvsidelang{+33 781 699 946}{+33 781 699 946} \newline
    \iconl{\faCar}\small        \cvsidelang{Permis B}{Driving License} \newline

    \bigskip
    
%................................................................
%
%                               Paracol -- Languages
%_______________________________________________________________
    
    \infobox{\cvboxlang{Langues}{Languages}} \\[0.5em]
    
    \begin{itemize}[leftmargin=*]
        %\item[\ding{120}] \cvsidelang{Français: Langue native}{French: Native}
        \item[\ding{120}] \cvsidelang{Français}{French}
        %\item[\ding{120}] \cvsidelang{Anglais: TOEIC 820}{English: TOEIC 820}
        \item[\ding{120}] \cvsidelang{Anglais}{English}
    \end{itemize}
    
    \bigskip

%................................................................
%
%                               Paracol -- Skils
%_______________________________________________________________

    \infobox{\cvboxlang{Compétences}{Skils}} \\[0.5em]
    \begin{itemize}[leftmargin=*]
        \item[\ding{120}] \cvskil{Modern C++ | C++20}
        \item[\ding{120}] \cvskil{Embedded C}
        \item[\ding{120}] \cvskil{git}
        \item[\ding{120}] \cvskil{bash | python | lua}
        \whenSubjectEmbedded
        {
            \item[\ding{120}] \cvskil{STM32 | stlink | gdb-multiarch}
        }
        \item[\ding{120}] \cvskil{gdb | sanitizer | clang-format/tidy}
        \item[\ding{120}] \cvskil{Latex | Markdown}
        \item[\ding{120}] \cvskil{Linux | Windows | WSL}
    \end{itemize}

    \medskip
    
    \infobox{\cvboxtext{Build / Remote / CI}} \\[0.5em]
    \begin{itemize}[leftmargin=*]
        \item[\ding{120}] \cvskilv{Bazel}{Assets/SkilsLogo/bazel.svg}
        \item[\ding{120}] \cvskilv{Buildkite}{Assets/SkilsLogo/buildkite.svg} | \cvskilv{BuildBuddy}{Assets/SkilsLogo/buildbuddy.svg}
        \item[\ding{120}] \cvskilv{Github Workflows}{Assets/SkilsLogo/github.svg} | \cvskil{Gitlab CI}
        \item[\ding{120}] \cvskil{Docker}
        \item[\ding{120}] \cvskilg{Premake}{Assets/SkilsLogo/premake.png} | \cvskil{CMake}
    \end{itemize}

    \medskip

    \infobox{\cvboxlang{Outils}{Tools}} \\[0.5em]
    \begin{itemize}[leftmargin=*]
        \item[\ding{120}] \cvskilv{VSCode}{Assets/SkilsLogo/vscode.svg} | \cvskil{Visual Studio}
        \item[\ding{120}] \cvskilv{Github}{Assets/SkilsLogo/github.svg}
        \item[\ding{120}] \cvskil{Jira | Bitbucket}

        \whenSubjectEmbedded
        {
            \item[\ding{120}] \cvskil{STM32CubeMX | PlatformIO}
            \item[\ding{120}] \cvskil{PCB: KiCad}
        }
        
        \item[\ding{120}] \cvskil{CAD: Fusion360}
    \end{itemize}
    
    \bigskip

%................................................................
%
%                               Paracol -- Hobbies/Associative
%_______________________________________________________________

    \infobox{\cvboxlang{Centres d'intérêt/Associatif}{Hobbies/Associative}} \\[0.5em]
    
    \begin{itemize}[leftmargin=*]
        \item[\ding{120}] \cvsidelang{Aréomodélisme | Drone FPV}{Aeromodelling | FPV Drone}
        \item[\ding{120}] \cvsidelang{FabLab}{FabLab | DIY for Escape-game}
        \item[\ding{120}] \cvsidelang{Électronique}{Electronics | DIY}
        \item[\ding{120}] \cvsidelang{Impression 3D | CNC}{3D Printing | CNC}
    \end{itemize}

    \bigskip
    
\end{flushleft}

\bigskip

%................................................................
%
%                               Paracol -- Close
%_______________________________________________________________

\switchcolumn

%................................................................
%
%                               No Header Init
%_______________________________________________________________

\ifheader
\else
    \Huge Sacha BELLIER
    
    \bigskip
    
    \large \cvmainlang
    {
        Master 2 - Ingénieur Informatique \whenSubjectEmbedded{Embarqué}\whenSubjectGraphics{Bas-niveau}
    }
    {
        Master 2 - \whenSubjectEmbedded{Embedded}\whenSubjectGraphics{Low-Level} Software Engineer
    }
    \medskip
\fi

%................................................................
%
%                               Profil
%_______________________________________________________________

\small \section*{Profil}
\cvmainlang
{
\ProfilFR
\whenAvailableLoc{\textbf{Disponible sur toute la France et le Monde}.}
}
{
\ProfilEN
\whenAvailableLoc{\textbf{Available across France and the world}.}
}

\medskip

%................................................................
%
%                               Professional experiences
%_______________________________________________________________

\section*{\cvmainlang{Expériences professionnel}{Professional experiences}}
\begin{tabular}{r| p{0.55\textwidth} c}

    \cvexppro
        {2024}
        {Thales}
        {\cvmainlang{Simulation d'asservissement}{Simulation of Guidance and Control}}
        {\cvmainlang{Stage de fin d'étude}{End-of-Studies Internship}}
        {Élancourt}
        {
            \cvmainlang
                {
                    Projet en \textbf{C++14} servant de banc de simulation. Génération de trajectoires réalistes dans un repère \textbf{WGS84} avec la librairie \textbf{Eigen}.
    
                    \vspace{0.4em}
                    \textbf{Initiative}: Mise en place et présentation de \textbf{Bazel} avec \textbf{Remote Build (bazel-buildfarm)}.
                }
                {
                    A \textbf{C++14} project used as test bench. Generation of realistic trajectories, using the \textbf{Eigen} library, in the \textbf{WGS84} coordinate system.
    
                    \vspace{0.4em}
                    \textbf{Initiative}: Setup and presentation of \textbf{Bazel} with \textbf{Remote Build/Cache} on \textbf{Bazel-BuildFarm}.
                    \ifVerbose
                    
                    Integration with RedHat's devtools and the project's own toolchain.
                    \fi
                }
        }
        {Assets/Compagny/thales.png} \vspace{1.2em} \\
        
    \cvexppro
        {2023}
        {KRONO-SAFE}
        {\cvmainlang{Contrôle en temps réel d'un bras Robot}{Real-time Control of a Robot Arm}}
        {\cvmainlang{Projet de fin d'étude}{End-of-Studies Project}}
        {Paris}
        {
            \cvmainlang
                {
                    Projet basé sur une carte \textbf{Zynq UltraScale+}. Développé en PsyC, un langage temps réel. Création de \textbf{Fenêtres de temps} validées par simulation afin de garantir le \textbf{temps réel} et le \textbf{déterminisme}.
                    
                    Développement d'un \textbf{driver I2C} pour interfacer une carte d'extension PWM. Réception de commandes Gcode via \textbf{UART}.
                }
                {
                    Project based on a \textbf{Zynq UltraScale+} development board. Developed in PsyC, a real-time language. Creation of \textbf{Timing windows} validated through simulation to guarantee \textbf{real-time} and \textbf{determinism}.
                    
                    Development of a custom \textbf{I2C driver} to interface a PWM extension card. Receiving Gcode commands via \textbf{UART}.
                }
        }
        {Assets/Compagny/krono-safe.jpg} \vspace{1.2em} \\

    \cvexppro
        {2022}
        {CNRS, Digital Holography Foundation}
        {\cvmainlang{Holographie oculaire en temps réel, CUDA}{Real-time eye Holography, CUDA}}
        {\cvmainlang{Stagiaire - 6 mois}{Intern - 6 months}}
        {Paris}
        {
            \cvmainlang
            {
                Projet en \textbf{C++20} et \textbf{CUDA} pour calculer un hologramme en temps réel.
                Utilisation de calculs parallèles et d'exécutions asynchrones.
                
                L'interface a été faite en \textbf{Qt4}.
            }
            {
                A \textbf{C++17} and \textbf{CUDA} program to process an hologram in real-time.

                Use of parallel computing and asynchronous execution.
                
                The UI was made using \textbf{Qt4}.
            }
        }
        {Assets/Compagny/CNRS.jpg} 

\end{tabular}

\medskip

%................................................................
%
%                               Projects
%_______________________________________________________________

\ifVerbose
\else

\section*{\cvmainlang{
        Projets personnel \hspace{.4em} (\faGithub \hspace{.18em} \linktext{https://github.com/0-Sacha}{0-Sacha})
    }{
        Personnal projects \hspace{.4em} (\faGithub \hspace{.18em} \linktext{https://github.com/0-Sacha}{0-Sacha})}
    }
    
\begin{tabular}{>{\small\bfseries}l c >{\small}p{0.85\textwidth}}
    
    \cvmaintext{\sellang{Entity Component System}{Entity Component System}}
        & (\linktext{https://github.com/0-Sacha/LittleECS}{\sellang{lien}{link}})
        & | \textbf{C++20 templates/concepts}, \textbf{Bazel}, \textbf{CI} \vspace{.08em} \\

    \cvmaintext{\sellang{Logger \& Sérialiseur Générique}{Templated String Serializer \& Logger}}
        & (\linktext{https://github.com/0-Sacha/ProjectCore}{\sellang{lien}{link}})
        & | \vspace{.8em} \\


    \cvmaintext{\sellang{Simple Moteur de Jeu}{Simple Game Engine}}
        &  
        & C++20, glm, OpenGL, GLFW, ImGui \vspace{.08em} \\

    \cvmaintext{\sellang{RayTracer sur CPU}{CPU-based RayTracer}}
        &  
        & C++20, glm, Vulkan, ImGui \vspace{.8em} \\


    \cvmaintext{\sellang{Rédaction: IMU et Kalman Filter}{Redaction: IMU and Kalman Filter}}
        & (\linktext{https://github.com/0-Sacha/mpu6050_kalman}{\sellang{lien}{link}})
        & | Embedded C/C++20, STM32, \textbf{Eigen} \vspace{.08em} \\

    \cvmaintext{\sellang{Self-balancing robot}{Self-balancing device}}
        & 
        & | \vspace{.08em} \\
        
\end{tabular}
\fi

%................................................................
%
%                               Degree
%_______________________________________________________________

\section*{\cvmainlang{Formation}{Degree}}
\begin{tabular}{r p{0.4\textwidth} c}

\cvdegree
    {2019 - 2024}
    {\cvmainlang{Master - Ingénieur en informatique}{Master's Degree - Computer Engineering}}
    {Epita, Paris, France}
    {0.9cm}
    {Assets/Epita.png} \vspace{1.2em} \\

\cvdegree
    {2021}
    {\cvmainlang{Semestre à l'étranger Erasmus}{Erasmus semester abroad}}
    {\cvmainlang{Vilnius Gediminas Technical University, Vilnius, Lithuanie}{Vilnius Gediminas Technical University, Vilnius, Lithuania}}
    {0.6cm}
    {Assets/Vilnius.png}
        
\end{tabular}

\medskip

%................................................................
%
%                               Final Footer
%_______________________________________________________________

\begin{center}
\small
\icon{\faPhone}{cvcolor_final_icon}{} +33 781 699 946
\icon{\faAt}{cvcolor_final_icon}{} {bellier.sacha2@gmail.com}
\end{center}

\newpage

\end{paracol}

%................................................................
%
%                               Projects Page
%_______________________________________________________________

\ifVerbose
    
    \restoregeometry
    
    \newgeometry{
        total={170mm,257mm},
        left=20mm,
        top=20mm,
    }
    
    \section*{\cvmainlang{
        Projets personnel \hspace{.4em} (\faGithub \hspace{.18em} \linktext{https://github.com/0-Sacha}{0-Sacha})
    }{
        Personnal projects \hspace{.4em} (\faGithub \hspace{.18em} \linktext{https://github.com/0-Sacha}{0-Sacha})}
    }
    
    \begin{tabular}{>{\small\bfseries}l p{0.8\textwidth} c}
    
        \cvmaintext{
        \linktext{https://github.com/0-Sacha/LittleECS}{
            \sellang{Entity Component System}{Entity Component System}
        }} &
            \textbf{C++20 requires/concepts}
            \newline Template-Metaprogramming / Static, Dynamic Polymorphism
            \newline C++ Iterators
            \newline \textbf{Bazel}: Build / Test
            \newline \textbf{CI: Github Workflows, Buildkite, and Buildbuddy} 
            \newline \textbf{Remote Build: Buildkite and Buildbuddy}
            \vspace{1.4em} \\

        \cvmaintext{
        \linktext{https://github.com/0-Sacha/ProjectCore}{
            \sellang{Sérialiseur Générique}{Templated String Serializer}
        }} &
            \textbf{C++20 requires/concepts}
            \newline Template-Metaprogramming / Static Polymorphism
            \newline Design Patterns (Factory, CRTP, Strategy, ...)
            \newline \textbf{Bazel}: Build / Test
            \newline \textbf{CI: Github Workflows, Buildkite, and Buildbuddy}
            \newline \textbf{Remote Build: Buildkite and Buildbuddy}
            \vspace{1.4em} \\
    
    \whenSubjectEmbedded
    {
        \cvmaintext{
        \linktext{https://github.com/0-Sacha/mpu6050_kalman}{
            \sellang {Rédaction: IMU et Kalman Filter}{Redaction: IMU and Kalman Filter}
        }} &
            \textbf{Embedded C / C++20} on \textbf{STM32}
            \newline \sellang{\textbf{Filtre Kalman} réalisé sur Eigen}{\textbf{Kalman Filter} made on Eigen}.
            \newline \textbf{I2C} to interact with an \textbf{IMU}; \textbf{UART} for telemetry.
            \newline Hardware Timer based software.
            \vspace{1.4em} \\
            

        \cvmaintext{
            \sellang {Self-balancing robot}{Self-balancing robot}
        } &
            \textbf{Embedded C / C++20} on \textbf{STM32}
            \newline \sellang{\textbf{Contrôleur PID} et Filtres (\textbf{Kalman}, Low Pass) réalisés sur Eigen}{\textbf{PID controller} and Filters (\textbf{Kalman}, Low Pass) made on Eigen}.
            \newline \textbf{I2C} to interact with an \textbf{IMU}.
            \newline Hardware Timer / Interrupt based software.
            \newline \textbf{Bazel}: Build \sellang{sur une toolchain}{with an} \textbf{arm-none-eabi} \sellang{}{toolchain}
            \vspace{1.4em} \\

        \cvmaintext{
           CPU Raytracer / Game Engine
        } &
            \textbf{C++20}
            \newline OpenGL / GLSL, ImGui, GLFW
            \newline math \sellang{sur}{using} glm
            \newline Design Patterns (Abstract Factory, Strategy, Singleton, ...)
            \newline \sellang{Compilé à l'aide de Visual Studio et Premake}{Compiled using Visual Studio and Premake}.
            \vspace{1.4em} \\
    }

    \whenSubjectGraphics
    {
        \cvmaintext{
        \linktext{https://github.com/0-Sacha/RayTracer}{
            \sellang {Raytracer sur CPU}{CPU Raytracer}
        }} &
            \textbf{C++20}
            \newline ImGui, Vulkan
            \newline math \sellang{sur}{using} glm
            \newline \sellang{Compilé à l'aide de Visual Studio et Premake}{Compiled using Visual Studio and Premake}.
            \vspace{1.4em} \\

        \cvmaintext{
        \linktext{https://github.com/0-Sacha/Blackbird}{
            \sellang {Moteur de Jeu}{Game Engine}
        }} &
            \textbf{C++20}
            \newline OpenGL / GLSL, ImGui, GLFW
            \newline math \sellang{sur}{using} glm
            \newline Design Patterns (Abstract Factory, Strategy, Singleton, ...)
            \newline \sellang{Compilé à l'aide de Visual Studio et Premake}{Compiled using Visual Studio and Premake}.
            \vspace{1.4em} \\
    }

    \end{tabular}
    
        
    \whenAcademicProjects
    {
        \section*{\cvmainlang{Projets académique}{Academic projects}}
        
        \begin{tabular}{>{\small\bfseries}l p{0.8\textwidth} c}
        
            \cvmainlang{Chiffrement sur STM32 (ARM)}{Encryption on STM32 (ARM)} &
                \textbf{Embedded C} on an \textbf{STM32}
                \newline \sellang{Chiffrement utilisant}{Encryption using} \textbf{mbedTLS}.
                \newline \sellang{Interfaçage avec un script Python / communication UART}{Interfacing with a Python script / UART communication}.
                \vspace{1.4em} \\
    
            \cvmainlang{Synthèse d'un ARM7TDMI}{Synthesis of an ARM7TDMI} &
                \textbf{VHDL} \sellang{sur un}{on an} \textbf{Altera Cyclone}
                \newline ModelSim, Intel Quartus Prime
                \newline ARM CPU architecture (CPU pipeline / ALU / memory, ...) 
                \vspace{1.4em} \\
    
            \cvmainlang{FreeRTOS Software PWM}{FreeRTOS Software PWM generation} &
                \textbf{Embedded C} \sellang{sur une carte de développement STM32}{on an STM32 development board}.
                \newline \sellang{Génération logicielle de signaux PWM sur \textbf{FreeRTOS}}{Software PWM generation using \textbf{FreeRTOS}'s Tasks}.
                \newline \sellang{Télémétrie sur un écran LCD connecté via SPI}{Telemetry on an LCD connected via SPI}
                \vspace{1.4em} \\
    
            \cvmainlang{Driver Linux pour un lecteur RFID}{Linux Driver for an RFID cards reader} &
                C89
                \newline Regmap API
                \newline \sellang{Interface SPI entre un Raspberry Pi et un lecteur de cartes RFID}{SPI interface between a Raspberry Pi and an RFID cards reader}
                \vspace{1.4em} \\
    
            \cvmainlang{Compilateur}{Compiler} &
                Compiler for my school's internal language (similar to Ada)
                \newline Parsing \sellang{utilisant}{using} Bison / Flex
                \newline LLVM AST / Visitor Design Pattern
                \vspace{1.4em} \\
    
            \cvmainlang{Linux temps-réel avec Yocto}{Yocto with real-time Linux} &
                \textbf{PREEMPT\_RT} patch \sellang{pour}{for} Linux
                \newline \sellang{Ajout et patch de jeux rétro à la distribution Yocto}{Add and patch retro games to the Yocto distribution}.
                \vspace{1.4em}
                    
        \end{tabular}
    }

\fi

\end{document}
